\documentclass[a4paper,10pt]{article}
\usepackage[utf8]{inputenc}
\usepackage[UKenglish]{babel}
\usepackage{fancyhdr}
\usepackage{anysize}
\usepackage{amsmath, amsthm, amssymb}
\usepackage{lastpage}
\usepackage[all]{xy}  % drawings
%\usepackage{listings} % code highlighting
\usepackage[usenames,dvipsnames]{color}
\usepackage{graphicx}
\usepackage{caption}
\usepackage{subfigure}

\pagestyle{fancy}
\fancyfoot[R]{\em \thepage / \pageref{LastPage}}
\fancyfoot[C]{}
\fancyfoot[L]{\em Master VIBOT}
\fancyhead[R]{\em Lab2 - Line features with Split \& Merge}
\fancyhead[C]{}
\fancyhead[L]{\em Probabilistic Robotics}
\renewcommand{\headrulewidth}{0.4pt}
\renewcommand{\footrulewidth}{0.4pt}

%\,	 a small space
%\:	 a medium space
%\;	 a large space
%\quad	 a really large space
%\qquad	 a huge space
%\!	 a negative space (moves things back to the left)
        
\begin{document}

\marginsize{2cm}{2cm}{2cm}{2cm}

% Title
%\hspace{1mm}
\begin{center}
\Large \textbf{Lab2 - Line features with Split \& Merge}
\end{center}
%\hspace{1mm}

\section{Introduction}

This lab is designed to learn how to use extract line features from point data provided as points $(x,y)$ in the plane, that have been projected from a \texttt{sensor\_msgs/LaserScan} in the bagfile dataset. The Split \& Merge algorithm explained in the attached scientific paper will be the base of its implementation.

The resulting algorithm will be used on the following labs as a base for feature extraction in Particle Filter, Extended Kalman Filter and Simultaneous Localization and MApping. 

\section{Pre-lab}

Read and understand the Split \& Merge algorithm explained in the attached paper. Instead of fitting a line to all points, we are going to implement the \emph{Iterative-End-Point-Fit} for its simplicity.

Lines will be represented in the form \eqref{line} because simplifies operations.

\begin{equation}
    a x + b y + c = 0 \label{line}
\end{equation}

\noindent
The questions for the prelab (answer in UdG moodle) are:

\begin{itemize}
    \item Given two points $(x_1, y_1)$ and $(x_2, y_2)$, compute the line parameters $(a,b,c)$.
    \item Given a line $(a,b,c)$ and a point $(x,y)$ compute the distance from the point to the line.
\end{itemize}

\noindent
Answer with a single formula containing only the given data.

\section{Lab work}

The folder named \texttt{splitandmerge} contains all necessary code for running this lab and a bagfile obtained from a turtlebot with \texttt{nav\_msgs/Odometry} and \texttt{sensor\_msgs/LaserScan} messages. Test the code by coping the folder in your workspace and running:

\begin{verbatim}
    roslaunch splitandmerge splitandmerge.launch
\end{verbatim}

As you can see the dummy Split \& Merge algorithm implemented, only provides lines from the first to the last point of each scan without any other computation. Also you can observe how
the odometry is not perfect and the accumulation of the scans does not represent properly the map (Fig.~\ref{map}).

\begin{center}
	\includegraphics[width=0.50\textwidth]{lab2-scans}
	\captionof{figure}{All points plotted with respect to odometry.}
	\label{map}
\end{center}

For completing the lab you have to modify the file under \texttt{src} folder called \texttt{splitandmerge.py}. The functions that have to be modified are highlighted.

For viewing properly each scan in Rviz, decrease the \texttt{LaserScan} decay time to 0, so it will only show the last published scan. Also change the fixed frame to \texttt{openni\_depth\_frame} to see the position of the robot fixed.

\section{Optional}

Represent all the lines that have been obtained by the Split \& Merge algorithm in a single image, showing the discrepancies caused by the odometry.

\section{Lab report}

Write a brief report (maximum 2 pages) explaining your solution and problems faced. Include the final \texttt{splitandmerge.py} file.

\end{document}
